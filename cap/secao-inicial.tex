\chapter{Objetivos}

%Este estudo visa pesquisar a viabilidade técnica e as considerações a serem efetuadas para a utilização do CUAD em empreendimentos de infraestrutura.

%Apresentar os conceitos inovadores do concreto de ultra alto desempenho (CUAD)  e compará-lo com outra solução usual para projetos de pontes de concreto armado convencionais, utilizando, quando possível, índices de custo, produtividade e racionalização da obra, de modo mostrar como essas informações podem influenciar a escolha da solução pelo projetista. Também pretende-se demonstrar o aspecto sustentável do material, citando o tipo de encomia gerada e como seu processo inovador de manufatura é mais ecológico.

%Apresentar os conceitos inovadores do concreto de ultra alto desempenho (CUAD), destacando suas vantagens e desvantagens. Mostrar como o índice de custo, produtividade e racionalização da obra podem influenciar a escolha da solução do projetista, levando em conta aspecto sustentável do material, citando o tipo de economia gerada. Apresentar obras executadas em CUAD e seus aspectos técnicos relevantes.

\nomenclature[A]{CUAD}{Concreto de ultra alto desempenho}

\section{Objetivo geral}

%Tem-se, por objetivo geral, apresentar e conhecer as características do CUAD, desde a sua fabricação até a sua moldagem e instalação no empreendimento.

Apresentar, conhecer e estudar o CUAD como material de construção civil e discutir a metodologia utilizada para sua aplicação como solução construtiva. 

\section{Objetivo específico}

%O objetivo específico deste trabalho é mostrar a viabilidade técnica da utilização de CUAD como material alternativo em grandes obras de engenharia, a partir das experiências internacionais com o material. 
%
%Abordar as limitações e requisitos necessários para a utilização do material, como deformação excessiva por conta da esbeltez e a necessidade da moldagem ser feita exclusivamente em um ambiente industrializado.

Apresentar um estudo de análise estrutural, onde, através de processos iterativos efetuado em um programa próprio (desenvolvido para ser executado no Scilab), analisa-se a possibilidade de se vencer grandes vãos com uma viga de CUAD, sem alterar sua seção transversal.

\chapter{Metodologia}

Para elaboração deste estudo, tomou-se como base referências bibliográficas, normas e recomendações técnicas, relatórios, artigos e pesquisas divulgadas no meio científico.

Por se tratar de um material relativamente novo e desenvolvido principalmente na Europa, a literatura disponível sobre CUAD	 em português é escassa, de modo que a revisão bibliográfica foi baseada em publicações de diversos países.

%Feita a coleta dessas informações, iniciou-se a pesquisa específica sobre a ponte estadunidense Jakway Park, por conta da sua execução em CUAD  e a sua seção inovadora, apresentando as características do projeto, e as considerações feitas para a sua execução.

Feita a coleta destas informações, iniciou-se a pesquisa específica sobre as propriedades do CUAD, necessárias para a correta modelagem e entendimento das propriedades físicas do material.

Coletou-se informações da ponte construída com CUAD Jakway Park, nos Estados Unidos, por conta da quantidade de informações disponíveis sobre o projeto e a direta colaboração de um dos engenheiros responsáveis pelo projeto com esta pesquisa. Por se tratar de um empreendimento executado em outro país, o trabalho fundamenta-se nos relatórios técnicos e científicos emitidos pelos engenheiros e pesquisadores responsáveis pelo projeto, já que a visita técnica não foi possível. Este projeto é de grande interesse para se entender os procedimentos necessários para se implantar uma solução construtiva com CUAD.

A pesquisa tem como foco o entender as possibilidades arquitetônicas e estruturais que podem ser atingidas com o CUAD, ao analisar a quanto pode-se aumentar vão teórico de viga protendida sem aumentar sua seção transversal, bastando aumentar a resistência característica à compressão do concreto.

A parte prática desta pesquisa fundamenta-se principalmente na NBR 6118:2014, no livro publicado por \citeonline{Cholfe} e nas recomendações para projetos com CUAD publicadas pela \citeonline{AFGC}.

%Desse modo, a pesquisa tem como foco a condições necessárias para o desenvolvimento e execução um projeto em CUAD, desde as condições necessárias para sua modelagem computacional até sua instalação na obra.

%A parte prática desta pesquisa fundamenta-se principalmente nos documentos emitidos pelo Departamento de Transporte do estado de Iowa, EUA e pelo e-mail enviado pelo engenheiro Brian Keierbeler, o qual trabalhou diretamente no projeto da ponte Jakway Park.

%Foram realizadas pesquisas em artigos, revistas, livros e publicações das sobre CUAD, com ênfase no desempenho, propriedades do material, uso e aplicações.

%Escolheu-se uma ponte executada em CUAD para exemplificar suas aplicações, dificuldades e vantagens: a ponte Jakway Park, nos Estados Unidos.

%O engenheiro americano Brian Keierbeler forneceu dados e fotos sobre a ponte Jakway Park, os quais podem ser encontrados no \autoref{chap:jakway}.

%Para calcular a ponte de concreto armado utilizada no estudo de caso, foi utilizado o software T-Rüsch, desenvolvido pelos engenheiros Gustavo Elias Khouri, Mariana Silva Serapião e Sander David Cardoso, o qual utiliza as tabelas feitas por Rüsch para calcular os esforços em lajes de pontes.

%Por fim, comparou-se a ponte executada em CUAD com o projeto de concreto armado, tirando-se as devidas conclusões sobre as vantagens e desvantagens da escolha de cada método construtivo no exemplo em questão.


\chapter{Justificativa}

%Quando 30\% do investimento financeiro feito em uma obra é perdido em desperdício \cite{picchi} , assim como a ausência de processos otimizados (além do tradicionalismo inerente à indústria da categoria), vê-se uma grande janela de oportunidade para se
\citeonline{picchi} afirma que 30\% do investimento feito em uma obra é desperdiçado. Além disso, é importante notar que as obras no Brasil tem uma baixa adesão na adoção de processos otimizados na construção civil. Diante disso, vê-se a necessidade de
introduzir no mercado um produto que traga racionalização ao canteiro de obras, tenha um desempenho superior aos materiais disponíveis atualmente no mercado e que tenha um ciclo de produção e descarte mais sustentável e inteligente, agregando valor ao produto vendido ao cliente. No Brasil, existem poucos estudos publicados sobre CUAD e sua aplicação como material alternativo, tornando este tema de interesse para a pesquisa nacional. Além disso, existe grande curiosidade acerca deste material, já que produzir um concreto com resistências superiores a 150 MPa exige um controle de qualidade muito rigoroso.

Durante a pesquisa, aspectos importantes foram notados, principalmente em relação à viabilidade da implantação e manutenção de um empreendimento executado com CUAD, pois, apesar do seu custo inicial ser maior, a diminuição no volume de uso de concreto e a resistência superior do material (o qual exige muito menos manutenção do que outras soluções construtivas ao longo de sua vida útil) pode viabilizar sua escolha como solução construtiva.

%Por essa janela é que o CUAD pode passar e tomar de vez o lugar do concreto armado em obras de infraestrutura. Sua resistência à compressão pode ir de 150 MPa a 800 MPa e, devido ao ótimo empacotamento das suas partículas, o material apresenta baixa permeabilidade, baixa porosidade e maior resistência a ataques químicos.

%Para que todas estas características inovadoras possam ser aproveitadas ao seu máximo, é necessário pensar em novas formas e novas geometrias de construção, que podem ser extremamente diferentes do que os projetistas atuais estão habituados, principalmente se foram feitas com o auxílio de algoritmos que apresentem geometrias ótimas para determinado problema a ser resolvido.

%Há também o aspecto que o CUAD pode ser uma alternativa viável para a manutenção das pontes brasileiras. Com a mudança da NBR \nomenclature[A]{NBR}{Norma brasileira} 7188 (Carga móvel rodoviária e de pedestres em  pontes, viadutos, passarelas e outras estruturas) em 2013, que ocorreu devido ao aumento de tráfego nas rodovias brasileiras, muitas pontes poderão necessitar de uma readequação para atender à nova demanda rodoviária brasileira \cite{Silva}. Mais que isso: um levantamento feito pelo Tribunal de Contas da União (TCU) afirma que por volta de 75\% das pontes em rodovias do Brasil precisam passar por algum tipo de intervenção (recuperação, reforço ou alongamento estrutural), de modo a atender satisfatoriamente a demanda regional \cite{Silva}. Neste contexto, o CUAD pode se apresentar como uma solução eficiente e até mesmo mais rápida e barata que as usadas tradicionalmente.

%\chapter{Abrangência}

%Este trabalho introduziu o histórico, definição, composição, características, princípios e dosagem do CUAD, assim como seus aspectos sustentáveis e um breve panorama do material no mundo. Também foram mostradas algumas pontes executadas com este material inovador. As vantagens e desvantagens do uso do CUAD também foram discutidas.

%Ensaios de laboratório e impactos precisos no custo da obra não farão parte do escopo deste trabalho.

