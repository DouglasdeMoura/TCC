\chapter{Evolução do Concreto}

Demorou 20 séculos para o homem desenvolver, e transformar areia e calcário em verdadeiras pedras. Hoje, o concreto é o produto industrializado mais consumido no mundo. Por ser um material cerâmico, a suca matéria-prima existe em praticamente todo mundo.
As propriedade de um concreto está ligada principalmente por ser um material versátil, durável e de bom desempenho. Dando uma boa vida útil a construção. Frente a outros materiais o concreto possui bastante competitividade tanto por suas características técnicas, como pelo viés econômico.
Embora para a fabricação do Cimento (matéria-prima do concreto) é emitido uma boa quantia de gases poluentes, ainda assim o concreto se enquadra muito bem na definição de sustentabilidade. Isso ao compararmos as escalas de uso do concreto com outro material construtivo.
Neste presente capítulo é abordado a história do concreto, desde que este era usado somente como material aglomerante natural, até as grandes construções específicas, que se utiliza concreto especial.

\section{Primeiros indícios}

O uso de materiais para construção está difundido com a própria história do Homem. Pois para as necessidades básicas e sobreviência o homem teve que fazer uso do que se encontrava na natureza. Segundo Cohein (1988) o primeiro material a ser utilizado pelo homem intencionalmente, foi a argila. Foi quando o homem pôde moldar uma pedra e outros utensílios domésticos.
Segundo Malinowski e Garfinkel (1991), logo depois da argila o homem começou fazer uso de cal e gesso, tendo como proposito revestir paredes e pisos. Por volta de IX e VII a.C. há indícios na cidade de Jericó, sudeste d Galiléia, pisos com estes materiais. Levando a xrer que a invenção do concreto surgiu com os romanos ou gregos.

\subsection{Gregos}

Na Grécia os principais indícios dos materiais usados está nos revestimentos de fontes atenienses, como a cal hidráulica, no começo do século V a.C. (ISAIA, 2011).
Os gregos usavam muito as pedras. O Partheron é uma das edificações mais conhecidas, feita exclusivamente de mármore branco. Mas em locais que não haviam Jazaidas de mármore, os gregos mostraram domínio surpreendente no concreto. Com um conhecimento pouramente empírico fizeram construções de edifícios e obra de infraestrutura. Em um estudo de caso relatado por Koui e Ftikos (1998), uma cisterna de concreto para armazenar água de chuva, em Kamiros, ilha de Rodes, Grécia. Foi pego dela uma amostra do material utilizado, e a partir de análises em laboratório feitos por Koui e Ftikos (1998) revelearam a qualidade surpreendente do material. O traço do concreto que eles fizeram mostrava uma curva granulométrica que se aproxima da curva ideal de Fuller. O material utilizado pelos gregos foram o seixo rolado, agregado calcário médio e fino, terra vulcânica e cal. A resistência a compressão atingifa por esse concreto é de 13,5 MPa, o que é inacreditável, por se tratar de uma material feito há 27 séculos atrás.

\subsection{Romanos}

Ao contrário dos gregos que mais utilizavam pedras para fazer vigas e colunas, os romanos preferiram o concreto. Assim sendo destacaram em construções grandiosas com amplos espaços de abóbadas e cópulas altas e dimensões enormes. Nesta época o curso de Arquitetura e Engenharia ganhava notoriedade para estudos.
Vitrúvio (1 a.C.) cita a composição do concreto em seu livro De Archtectura, sendo algira caulinítica calcinada ou pedra vulcânicas calcinadas, areia vulcânica reativa de origem natural. Importante ressaltar que eles usavam um traço para ambiente aquático e outro para ambiente seco. Sendo empregado pedras vulcânicas quando era feito portos e pilares de ponte, aqueduto. Substituindo assim a areia vulcânica. Esta, era usada principalmente onde não havia muito movimento de água.
Com o estudo de acertivas e erros deste material, os romanos conseguiram criar alguns métodos construtivos. Alguns somente como mudança de Layout, outros como reais soluções arquitetônicas, como é o caso das abóbadas e cópulas.

\susubsection{Panteão Romano}

Dentre as muitas obras de extrma relevância que os romanos executaram, o Panteão é em si a mais desafiadora para engenharia da época. Com um vão livro de 43,30m, o Panteão ficou por 18 séculos sendo o maior vão do mundo (ISAIA, 2011). Construído por Agripa em 27 a.C. e destruído pelo fogo, e reconstruído por Adriano por volta de 120 d.C., o Panteão comça em sua base das paredes com 6 metros de espessura, e o seu teto abobadado de concreto, sendo cuidadosamente elaborado com materiais leves, com espessuras pequenas e baixa densidade de agregados. Em sua maior parte, compõe o mármore e carvão, depois tijolos quebrados, e por fim pedra pome no topo. (MOORE, 1995)

\section{Idade moderna}

A idade média foi um perído que não teve avanços no tocante da evolução do concreto, mas teve um princípio para as explicações dos porquês, que os romanos outrora não entendiam.
Marcando o fim da idade antiga, a queda do império romano em 476 d.C. também marcou o fim dos avanços tecnólogicos. Foi levado ao esquecimento o modo de vida, a cultura e a tecnologia que os romanos tinham, Mas o surgimento do Renascimento, marcando o início da era moderna, por volta do século XIV, o homem retomou as referências da antiguidade clássica (ISAIA, 2011).
É onde hpa a explosão de personalidades ímpares, como Leonardo da Vinci (deu forma as ideias de Vitrúvio apresentada no livro De Architectura), Michelangelo, e outros figurões. Uma grande referência da época é Galileo Galilei, este tem sua fundamental importância para o concreto, pois ele quem começou a dar bases para a teoria da engenharia das vigas.
O filósofo e Arquiteto Robert Hooke, desenvolveu teoria sobre tensão x deformação dos materiais, onde posteriormente seria conhecida como Lei de Hooke.
Newton posteriormente, desenvolveu as leis do movimento, e Leibniz, dando fundamento ao cálculo, obteve-se então pela primeira vez a compreensão das leis fundamentais que regem as estruturas.

\section{Idade contemporânea}

Aqui os cálculos foram aprimorados, e a revolução industrial colaborou muito para a história do concreto de hoje. A produção de navios e principalmente locomotivas, se viu necessário superar algumas barreiras geográficas, elaborando pontes e ferrovias.
O ferro fundido foi a saída para construção das pontes. Com a produção em massa de ferro fundido em perfil, barras e grampos a construção civil viu a possibilidade de diminuir a espessura de paredes, pilares e vigas com o ferro. O panteão de Paris foi a primeira construção a utilizar um método construtivo totalmente novo, possibilitando maiores janelas e maiores vão. Essa construção foi inovadora, e possibilitou maiores vãos pois Rondelet, arquiteto teórico, fez essa obra baseada totalmente em seus cálculos, cálculos esses, com princípios aos que hoje são usados. Neste método construtivo ele adotou grampos de ferro fundido entre a alvenaria. Os cálculos de Rondelet estavam corretos, entretanto apareceram fissuras e assentamento a edfificação, causada principamente pela fluência do material, que não foi considerada na época.

\subsection{Cimento Portland}

Desde o início da idade contemporânea houve-se uma busca por materiais que fossem mais resistentes e que competisse com o ferro. Foi-se elaborado alguns tipos de cimento, sendo muitos patenteados. Mas só em 1824 um construtor inglês chamado Joseph Aspdin que patenteou o que hoje conhecemos como cimento portland. Porém seu método era pouco preciso, pois os fornos para a produção do Clínquer (importante componente da composição do concreto), não era controlado. O que gerava um produto duvidoso. Só quando Aspdin passou o comando da empresa para os filhos, foi que começou a ter um maior controle no forno, garantindo uma qualidade melhor, e foi divulgado nesta mesma gestão, pesquisas que apontava o portland como resistência maior que o dobro da principal concorrência. Mas só em 1845 com o fim da patente de Aspdin que o "verdadeiro" inventor do cimento portland apareceu. O Isaac Johnson adquiriu a empresa dos filhos de Aspdin e produziu o Clínquer atravéz do alto forno (como é feito hoje) (GÓMEZ, 1973).
No Brasil a fabricação do Cimento chegou pelo comendador Antonio Proost Rodovalho em 1855 (BATTAGIN, 2010)

\subsection{Concreto e ferro}

O concreto já bastante disseminado na população no século XIX, a população começou utiliza-lo para outras finalidades além da construção civil. Lambot, um agricultor francês, que começou a utilizar o concreto com ferro, para fazer tanques. Em 1855, quando fez um barco de cimento com ferro, numa exposição em Paris, que patenteou esse método. Na exposição, o barco ganhou muita relevância, não por ser um barco, mas por ser armado com ferro. Mornier, jardineiro francês, observou aquele barco, e começou a produzir seus vazos com ferro e cimento. Ele evoluiu os métodos e começou produzir painéis para fachada, e em 1867, obteve uma patente desse método. Em 1875 construiu a primeira ponte de concreto armado do mundo, chamada Chazelet, com comprimento de 18,80m e largura de 4,25m, localizada na França.
Em 1884 uma empresa alemã se interessou pelo método, e comprou a patente de Monier para o norte da Alemanha, e outra empresa comprou para o resto da Alemanha. Porém em 1886 ambas empresas venderam a patente para Gustav Wayss, engenheiro alemão. Wayss foi quem dedicou tempo pra entender as funcionalidades e todos os conceitos do concreto armado. Buscava provas que o concreto resistia a compressão, enquanto o aço a tração. Fundou a Empresa Wayss & Freitag, com a empresa montada, ele foi o pioneiro no cálculo do concreto armado, e se expandiu bastante na época por ter a patente do melhor método construtivo que o homem já tinha visto. Até hoje ainda exite essa construtora.

\subsection{Desenvolvimento do concreto armado}

Desde o início do uso do concreto armado, até 1900, foi pouco utilizado, o conhecimento técnico profundo sobre esse material. Apenas Wayss investia em pesquisa. Mathias Koenen, arquiteto, foi contratado por Wayss, e assim teve o primeiro estudo profundo sobre o concreto armado, fazendo testes em laboratório, levando em consideração a elasticidade do material. Mas foi Mörsch quem aprimorou e estabeleceu bases científicas para o concreto armado, produziu um livro intitulado Der Betoneisenbau: Seine theorie und Anwendung (Construções de Concreto e Aço: Sua teoria e aplicação). Fez parte da empresa de Wayss por 34 anos, e lecionava na Universidade de Zurique.

A partir de 1921, começou na França a vibrar o concreto pra diminuir os vazios em seu inteiror. E foi quando os conceitos do concreto protendido começaram a ser elaborados. Porém só em 1946 que a primeira estrutura em concreto protendido foi executada. uma ponte sobre o rio Marne, na França. Sendo o maior recorde de vão livre da época, 180 metros.

Depois de 1920 muitos foram os sistemas construtivos desenvolvidos, mas o marco maior foi uma ponte em arco tri-rotulado com 90 metros de vão, inaugurada em 1930 no vale de schiers, na Suíça. Foi o Engenheiro Robert Maillart que projetou. Os projetos de Maillart que era notórios e essa ponte em arco é considerada Marco Histórico Internacional da Engenharia Civil pela Sociedade Americana de Engenharia Civil.

\section{Concreto no Brasil}

No Brasil desde 1907 a empresa Wayss & Freitag, construiu em São Paulo o primeiro prédio em concreto armado do Brasil, na rua São Bento. O prédio possuía apenas 3 pavimentos, mas isto já era um marco na época.

Na disseminação do concreto armado pelo Brasil, um nome deve ser levado em bastante consideração, o Emilio Henrique Baumgart. Tido como "pai do concreto armado no Brasil", formado em 1919 pela Poli RJ, estagiou os primeiros anos pós formado na empresa de Riedlinger (engenheiro alemão que trabalhava para Wayss), onde muito aprendeu (ISAIA, 2011), Ainda estagiário, ajudou no projeto estrutural do hotel Copacabana Pallace, no Rio de Janeiro. Depois de 4 anos que se formou, decidiu seguir carreira solo. Nesta nova etapa da sua vida, projetou mais de 300 edifícios, 100 pontes, 80 conjuntos industriais e mais de 500 projetos menores (VASCONCELOS, 2005). Muitas obras notáveis ele fez, umas delas chegou a ser destaque mundial por quebrar recordes, são elas: o edifícil A noite - maior do mundo em 1928; A ponte ferroviária sobre o rio Mucuri em 1939, com 39,30m de vão principal e 142,60m de comprimento, sendo que anteriormente a maior ponte tinha em seu comprimeiro total 51m e maior vão 17m.

\subsection{Brasília}

O projeto de Brasília foi possível graças a três grandes nomes, o primeiro deles Lúcio Costa, urbanista de referência nacional, projetou o plano piloto de Brasília. Oscar Niemeyer, um dos maiores arquitetos e urbanista em escala mundial. E joaquim Cardozo por fazer com que as ideias de Niemeyer pudessem ser construídas.
Todos os Edifícios de Brasília, exceto os 12 blocos dos ministérios, foram constituidos de concreto armado ou protendido. A frase mais marcante de Niemeyer é "não é o ângulo reto que me atrai, nem a linha reta, dura, inflexível criada pelo homem. O que me atrai é a curva livre e sensual, a curva que encontro na montanha do meu país, no curso sinuoso dos seus rios, nas ondas do mar, no corpo da mulher preferida. De curvas é feito todo universo, o universo de Einstein". Daí todas as curvas encontradas em Brasília, onde entra também a genialidade do Joaquim Cardozo, que dividia todo esse vislumbre e fascínio pelas curvas, transgredindo algumas prerrogativas das Normas da época, e possibilitando a arte que é vista a céu aberto em Brasília.

\subsection{Grandes feitos da infraestrutura brasileira}

Como já citado Emilio Baumgart, o "pai do concreto armado no Brasil" foi o percursos de grandes feitos em território nacional. Mas muitas outras obras foi/é referência munidal, segundo Vasconcelos (1985) essas são as mais notórias:

	Elevador Lacerda, Salvadaor, 1930 - maior elevador do mundo na época, 72,5m de altura.
	Cristo redentor, Rio de Janeiro, 1930 - maios estátua em concreto pré moldado do mundo, 38m de altura.
	Ponte sobre o Rio das Antas, Rio Grade do Sul, 1952 - recorde mundial de ponte em arco, 186m de vão livre.
	Museu de Arte Moderna de São Paulo, São Paulo, 1968 - recorde mundial em viga reta protendida, simplesmente apoiada, 70 m de vão e 3,5m de altura.
	Edifícil Itália, São Paulo, 1965 - recorde mundial em altura, 151m, 46 pavimentos.
	Barragem Itaipu, São paulo, 1982 - maior hidréletrica do mundo em seu gênero, 196m de altura e 7919m de comprimento.
	Ponte Octávio Frias Filho, São Paulo, 2008 - única ponte estaiada do mundo com duas pistas curvas ligadas a um mesmo pilar em forma de X.

\section{Concreto hoje}

Depois de todos os avanços que o concreto passou até chegar no grau de ciência que o homem domina hoje, e por sua matéria prima ser tão vista, por precisar de uma mão de obra não especializada, não há local que não haja alguma edificação que não se utiliza o concreto. Tornando-se o material industrial líder em uso no mundo. Muito foi contribuido para isso, a evolução das ideias de Monier e outros cientistas, para tentar compensar a lacuna que o concreto puro deixa, por ter um resistência a tração baixa.

\subsection{Características}

O conjunto aço-concreto, por ocorrer ligações químicas que fortalece ainda mais um com o outro, garante uma boa resistência a compressão e a tração. Alinhando essas duas resistências há o melhor em questão de resistências a flexão, causando maiores vão livres com vigas retas ou curvas, ainda podendo aumentar esse valor utilizando o concreto protendido. As principais características do concreto é destacado a seguir:
	Disponibilidade de matéria prima: 89\% da crosta terrestre é formada oir 90\% dos materiais que é composto o concreto.
	Versatilidade na modelagem: no seu estado fresco é um material plástico, se adequando a qualquer forma que o projetista quiser.
	Sólido: depois da cura, se torna um amterial, praticamente homogêneo, solidariza entre si, chegando a uma rigidez desejada. 
	Durável: a partir de um bom traço, executado corretamente, tem uma boa resistência a agentes agressivos.
	Custo: por existir muito no ambiente e pelo bom desempenho, esse produto se mostra muito competitivo comparado com outros materiais.
	Sustentabilidade: Embora para sua produção utiliza fornos e emite quantidade significativa de gases poluentes, quando comparado com outro material construtivo, essa emissão é baixa. Além de não ter gastos com logística, transportando esse material por longo trexos, pois sua produção é local. E possibilita o uso de materiais reciclavéis em sua composição.
São muitos os pontos positivos, mas o concreto tem alguns negativos, como baixa resistência a tração, elevado peso próprio, suscetível a variação volumétrica, o calor gerado pela hidratação em peças volumosas. Enfim, esses pontos negativos podem ser eliminados ou minimizados. Assim, o concreto se mostra como um excelente material construtivo, tendo vantagens técnicas difíceis de ser superado por outro material.

\subsection{Concreto de alto desempenho (CAD)}

É predominante hoje o uso do concreto convencional, mas acredita-se que esse cenário será mudado com aditivos superplastificantes adicionados ao concreto, possibilitando menor quantia de água no preparo da massa, não comprometendo sua trabalhabilidade, diminuindo os vazios presentes no concreto. Esse elemento citado acima é um dos itens que pode ser feito para aumentar o desempenho do concreto.
Desde 1960 vem sendo divulgado pesquisas com propósitos de diminuir a relação água/cimento. Com a diminuição da água, há menos vazios no concreto, e isso afeta positivamente no fck, que é a resistência do concreto a compressão. Com o fck mais alto, há uma diminuição na dimensão das formas, menor quantidade de armadura e menos peso próprio. No The Sustainable Concrete Guide (2010) ao aumentar o fck de um concreto de 28MPa para 62MPa em um pilar de um prédio, há uma redução de 55\% no volume do concreto e dominuição dos gasto com concreto de 18\%.
No ano de 1988 foi inaugurado o primeiro prédio utilizando um concreto especial, na composição do concreto continha escória de alto forno e sílica ativa, superplastificante poderosos, que alcançaram um fck de 69MPa, o prédio consta com 68 pavimentos.Outro em Chicago, Estados Unidos da América, inaugurado em 1990, com 295m de altura e 65 pavimentos, o concreto utilizado em sua estrutura chegou a 83MPa, fazendo uso de cinza volante e sílica ativa (ISAIA, 2011). EM Kuala Lumpur, Malasia, foi construído em 1998 as torres gêmeas Petronas, são 88 pavimentos e 450m de altura. O fck do concreto utilizado é de 80MPa, Atualmente Petronas Tower é o 14º prédio mais alto do mundo. Em Taiwan, foi concluída em 2004 o edifícil Taipei com 509m de altura e 101 pavimentos. Até o 62º andar, ele foi construído com uma estrutura mista, em que foi colocado perfis metálicos de 8cm de espessura e injetado concreto de 69MPa. Atualmente é o 9º mais alto do mundo.
O mais alto do mundo, está localizado em Dubai, Emirados Árabes Unidos, é chamado de Burj Khalifa, 160 pavimentos e 828m de altura.
O centro empresarial das Nações Unidas e o edifícil e Tower, ambos localizados na zona sul de São Paulo, são os que tem maiores destaques em qualidade de concreto no Brasil. O Centro Empresarial tem 37 pavimentos e seu concreto tem uma resistência a compressão de 50MPa, e o edifícil e Tower utilizou um concreto de alto desempenho, atingindo um fck de 80MPa até o 4º andar.
O concreto de alto desempenho é mais requisitado hoje para ambientes expostos de bastante agente agressivos. Pois em sua microestrutura é muito bem compactado em relação ao concreto convencional, isso dificulta agentes agressivos infiltrar pelo concreto, evitando a corrosão da armadura.
\subsecion{Concretos específicos}
Tem-se hoje diversas demandas, cujo principal material que poderia ser atendido é o concreto, mas há algumas situações que o concreto convencional não teria um resultado satisfatório. Pensando em atender demandas específicas foi desenvolvidos alguns tipos de concreto especiais, são eles:
	Concreto com aditivos especiais: conforme descrito no item1.6.2, pode encontrar diversos casos em que foi adicionado algum aditivo no concreto para aumentar a sua resistência e outros o seu desempenho. O concílio de aditivos superplastificantes, possibilitou a criação de um concreto auto adensável, gerando economia de tempo e mão de obra, por não precisar ficar vibrando para compactar. Outros aditivos como o nitrito de cálcio, para combater a corrosão da armadura. E existem outros tipos que combate os efeitos da retração, outros para captar a água interna, e garantir a cura completa. Para cada propósito há possivelmente um tipo de aditivo.
	Concreto com fibras: As fibras no concreto tem o objetivo de aumentar a resistência a tração. Para este caso, a fibra age como pontes na pasta do concreto, distribuindo bem as cargas, consequentemente, minimizando e até evitando fissuras.
	Concreto Leve: Normalmente o concreto convencional tem peso próprio em torno de 2300kg/m³, para alguns casos, a fim de diminuir as cargas na fundação, utiliza-se concreto leve, onde, geralmente é obtido a partir de agregados leves naturais ou artificais.
	Concreto Decorativo: O concreto é a preferência entre arquitetos e engenheiros (ISAIA, 2011), por poder se transformar em qualquer forma, garantindo dupla função: estrutural e estética. Há no mercado concreto naturalmente branco, devido a um tipo especial de matéria prima, alcançando resistência similar ao concreto convencional.
	Concreto de pós-reativos: Com pesquisas profundas na microestrutura do concreto, foi identificado que a principal causa de ruptura em concreto se dava na fase do agregado a pasta. A interatividade entre esses dois causava uma deficiência na resistência final, sendo o ponto mais frágil. Ao analisar os dados, perceberam que quanto menor fosse o diâmetro do agregado, maior era a resistência do concreto. Então, em uma pesquisa francesa, concluíram que uma graduação de partículas menores que 3mm, conciliado com alto teor de finos compostos de cimento portland, aditivos seperplastificantes, fibras de alto desempenho, sílica ativa, pó de quartzo moído, essa mistura gera aumento da compacidade, em consequência o aumento da resistência mecânica. E por conter apenas particulas diminutas, surgiu o nome pó-reativos. Esse é um concreto considerado de Ultra Alto Desempenho (CUAD, ultrapassando um fck superior a 200MPa e em alguns casos a 500MPa.