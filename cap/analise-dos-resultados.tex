\chapter{Análise dos resultados}

O processo de pesquisa que resultou na realização deste trabalho permitiu as seguintes análises:

\begin{alineas}[label=\textbullet]
	\item Demonstrou-se a possibilidade da realização de empreendimentos e obras de arte no Brasil em CUAD, a partir dos fundamentos teóricos encontrados na literatura científica nacional e internacional, apesar de ainda não existir uma normatização de seu uso;
	\item Foi possível conhecer as características físicas do CUAD, a partir das quais é possível construir modelos computacionais que permitem a análise de estruturas feitas com o material;
	\item Compreender os benefícios do CUAD como solução construtiva em relação ao concreto convencional, CAD e aço, principalmente nas questões de manutenção da obra (menos manutenção ao longo da vida útil), sustentabilidade ambiental (menor impacto ambiental em seu ciclo de vida produtivo), tempo de execução (que é menor, por conta de seu processo de produção industrializado) e custos (que, apesar de ser maior no início, pode ser compensado ao longo da vida útil da construção, em sua manutenção e durabilidade);
	\item Foi possível conhecer, na revisão bibliográfica, as principais linhas de pesquisa que regem o desenvolvimento do CUAD nos principais centros de pesquisa do mundo, as marcas comerciais já disponíveis no mercado e outras variedades que estão sendo estudados e ainda não são vendidas;
	\item Foi possível conhecer as técnicas utilizadas na execução de estruturas em CUAD, principalmente na questão da distribuição das fibras no concreto e um método utilizado durante a mistura do CUAD utilizado para construir a ponte Jakway Park;
	\item Foi possível fazer um programa que possa auxiliar na análise da viabilidade da aplicação de CUAD em vigas protendidas, auxiliando o engenheiro a entender, com rapidez, a viabilidade da aplicação do material na obra;
	\item No estudo de caso, foi demonstrado que o uso de CUAD na construção de vigas protendidas pode diminuir a sua seção transversal em aproximadamente 50\%;
	\item No estudo de caso, comprovou-se a capacidade do CUAD de permitir a realização de empreendimentos com formas muito esbeltas, dada a sua elevada capacidade de resistência e sua qualidade inerente, fato que também se deve à sua produção industrializada.
\end{alineas}