\subsection{Componentes da mistura}
A utilização do concreto sempre tem como objetivo a obtenção de uma mistura com resistencia, trabalhabilidade e desempenho, sendo assim o desenvolvimento tecnologico do concreto passou a focar na otimização dos componentes que constituem a mistura, no modo de preparação e analise o objetivo final de sua utilização. Essa escala de analise faz com que o conceito de concreto fosse modificado em relaçaõ ao seu significado mais tradicional, passando a pensar no concreto como uma elemento de estético e não mais como um aglutinador de elementos.

As pesquisas realizadas desempenharam um papel de extrema importancia no conceito da construção, possibilitando a descoberta de concretos que pudessem desempenhar funções especificas de acordo com a necessidade exigida pela situção. Assim partiu-se de um concreto convencional, passando pelo concreto armado e chegando ao CAUD, que é definido pela ACI (1998) apud Tutikian (2011) como: um concreto que atenda uma combinação especial entre desempenho e requesitos de uniformidade que não pode ser atingida rotineiramente com uso de componentes concvencionais e praticas normais de mistura, lançamento e cura.

O principal diferencial que distingue o CUAD dos outros tipos de concreto está atrelado à justificativa de sua utilização, já que sua composição é basicamente a mesma em relação ao concreto convencional e aproveitando das mesmas materias-primas. Podendo essa justificativa estar relacionada ao meio em que o concreto estará exposto, a resistencia requerida, ao desemprenho ao longo do tempo ou até esbeltez da area ou perfil. E para que esse desempenho seja asseguradao necessitou-se de uma acompanhamento especifíco em todos os componentes da mistura como: fator agua/aglomerante; granulometria dos agregados miudos e graudos; adição de aditivos, processo de cura e adensamento.

Destrinchando os aspectos que influenciam diretamente o desenvolvimento do CAUD vimos que o fator agua/aglomerante esta intrinsicamente relacionado a resistencia mecanica, que seus agregados devem ser escolhidos de maneira cuidadosa para que a porosidade da mistura não seja afetada, a escolha dos aditivos é feita de maneira específica para cada finalidade, assim como os processos de manuseio, cura a e adensamento. Em essência, esses são os pontos básicos para uma mistura de CAUD, que sempre tem como objetivo primario uma compacidade maior entre seus componentes, para que sempre tenha como resultado uma pasta compctada com baixo teor de porosidade e microestrutura cerrada.

Para que esse resultado seja obtido necessita-se que: tenha uma diminuição do fator agua/aglomerante, relação agua/m³ e do ar aprisionado na massa, atraves de aditivos ou superplastificantes; escolha adequada dos agregados miúdos em relação a granulometria para uma maior compacidade, passando a ser mais fino que o proprio cimento em alguns casos; utilização de agregados graudos com menor diametro, isso faz com que essa parte do processo ganhe uma importancia elevada pois essa fase do concreto tende a ser a parte mais fragil da mistura, podendo comprometer a resistencia final; reforços das ligações quimicas entre as particulos atraves de adiçoes minerais que provocam o refinamento dos poros e grãos, como a silica ativa, metacaulim ou as cinzas de arroz, tambem conhecidas como superpozolanas. As consequencias diretas dessas ações resultam em uma microestrutura com poros de menor tamanho, resistencia a passagem de fluidos e maior capacidade de fixação de agentes dissolvidos. O resultado final dessa gama de processos nos leva a um aumento da compacidade, alta resistencia mecânica, elevação da durabilidade e de seu desempenho ao longo de sua vida util.

Assim evidencia-se que, além da busca por melhores materiais baseados na sua fabricação, temos que direcionar nossa atenção para as fases de agragados, zona de transição e pasta, entendenso que as fases possuem seu inicio na seleção de agregados e os processos subsequentes tem o começo inteiramente ligado ao termino do processo anterior. 

